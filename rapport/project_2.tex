\documentclass[a4paper,12pt]{article}
\usepackage{srcltx}
\usepackage[english,swedish]{babel}
\usepackage[utf8]{inputenc}
\usepackage[T1]{fontenc}
\usepackage{parskip}
\usepackage{amsmath}
\usepackage{amsfonts}
\usepackage{algorithmic}


\makeatletter
\def\imod#1{\allowbreak\mkern10mu({\operator@font mod}\,\,#1)}
\makeatother

\renewcommand{\O}{\ensuremath{\mathcal{O}}}
\renewcommand{\*}{\ensuremath{\cdot}}

\begin{document}

\selectlanguage{english}
\title{DD2440 Advanced Algorithms (fall 2009) \\ Project 2 -- Integer
factorization}
\author{Joel Pettersson \\ 880519-0637 \\ joelpet@kth.se \and Linus Wallgren \\
880213-0099  \\ linuswa@kth.se}
\date{\today}
\maketitle
\thispagestyle{empty}
\newpage
\setcounter{page}{1}

\selectlanguage{swedish}

\section{Inledning}

\subsection{Problembeskrivning}

Ett primtal är ett tal som endast är delbart med ett och sig själv. Alla tal som
inte är primtal går att dela upp i primtalsfaktorer. Det vi söker är varje sådan
primtalsfaktor till ett visst heltal, hädanefter refererat till som $n$ om inget
annat anges.




\section{Metod}




\subsection{Kvadratiska sållet}

Den grundläggande tanken bakom kvadratiska såll-metoden är relativt enkel. Målet
är att finna två tal $x, y \in \mathbb{Z}$ sådana att
\begin{align*}
    x^2 \equiv y^2 \imod{n}, \ \text{ men } x \not\equiv \pm y \imod{n}.
\end{align*}

Då råder enligt konjugatregeln följande ekvivalens
\begin{align*}
    n \mid (x^2 - y^2) \Leftrightarrow n \mid (x+y)(x-y).
\end{align*}

Vidare gäller det att 
\begin{align*}
    n \not\mid (x+y)  \  n \not\mid (x-y), 
\end{align*}
ty $x \not\equiv \pm y \imod{n}$.








\end{document}
